\documentclass[a4paper, 12pt]{article} %
\usepackage{graphicx,amssymb} %
\usepackage{url}			       % For \url
\usepackage{xcolor}
\usepackage[left=1cm, top=2cm, bottom = 2cm, right=1cm, nohead, nofoot]{geometry}
\usepackage{hyperref}

%\textwidth=15cm \hoffset=-1.0cm %
%\textheight=25cm \voffset=-1.5cm %

\pagestyle{empty} %

\date{1/10/2022} %

\def\keywords#1{\begin{center}{\bf Keywords}\\{#1}\end{center}} %

% Please, do not change any of the above lines

\tolerance=1
\emergencystretch=\maxdimen
\hyphenpenalty=10000
\hbadness=10000

\begin{document}

% Type down your paper title
\title{ASSESSING THE ECOLOGY OF THE FLINT RIVER IN FLINT, MICHIGAN ABOVE AND BELOW A CENTURY-OLD DAM}

\vspace{0.5cm}
% Authors
\author{Chloe Summers, Arianna Elkins, Cason Konzer, Heather Dawson  \\ %
       University of Michigan - Flint \\ % Affiliation 1
       % Add authors and affiliation as needed 
       \textit{ \color{violet}
       \href{mailto:summersj@umich.edu}{summersj@umich.edu} \ \href{mailto:arelkins@umich.edu}{arelkins@umich.edu} \ \href{mailto:casonk@umich.edu}{casonk@umich.edu} \ \href{mailto:hdawson@umich.edu}{hdawson@umich.edu}}  % Only one corresponding e-mail
       }%

\maketitle

\thispagestyle{empty}

% The abstract
%\vspace{2.5cm}

\begin{abstract}
\vspace{0.5cm}

       Habitat fragmentation is detrimental to biodiversity and productivity within an ecosystem. 
The Flint River Ecology Study aims to assess an area of the Flint River, tributary to Lake Huron, 
above and below a century-old dam in the City of Flint in Michigan prior to, during, and after dam removal and restoration efforts. 
We report on fish distribution, diversity, and perceived fish habitat within 250 meters of the terminal dam, up and downstream, 
collected prior to restoration efforts from May 2019-August 2019 and May 2021-November 2021. 
We used multiple gear types, and fish were identified, measured, weighed, and those 16 cm or larger were tagged with numbered floy tags. 
A subsample of fish were collected for contaminant testing. 
Bathymetry of the study site was determined and relevant abiotic data collected. 
Simpson’s diversity index for species richness above the dam was 0.762 and 0.671 below. 
We investigated fish feeding guilds above and below the dam and compared the guilds present with the habitat of the two areas. 
Mercury levels in non-piscivores were significantly higher in downstream fish, presumably because those fish were Great Lakes migrants. 
Sampling above versus below the dam enables comparisons within this ecosystem that investigates how habitat fragmentation impacts the ecology of the Flint River. 
This data will be used to compare future assessments once connectivity is restored.

\vspace{8cm}

% \color{violet}
% \textbf{Resource Management} : 
% \textit{Monitoring and Modeling Effects of Aquatic Barriers on River Ecosystems}
% \color{black}

\end{abstract}

\keywords{Flint River, Dam, Diversity} % Write down at least 3 Keywords

% \section{Introduction}

\end{document}